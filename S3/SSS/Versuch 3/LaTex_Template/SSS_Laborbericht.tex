%---------------
%╔═╗╔═╗╔╦╗╦ ╦╔═╗
%╚═╗║╣  ║ ║ ║╠═╝
%╚═╝╚═╝ ╩ ╚═╝╩
%---------------

% language setup
\newcommand{\docLanguage}{ngerman}
%\newcommand{\docLanguage}{english}

% DOCUMENT SETUP
\documentclass[12pt, oneside, a4paper, \docLanguage]{report}
\usepackage[left=3cm,
			right=2.5cm,
			top=2.5cm,
			bottom=2.5cm,
			includehead,
			includefoot]{geometry}

% line spacing
\usepackage{setspace}
\setstretch{1,25} % 15/12 --> 1.25

% encoding setup
% T1 font encoding for languages that use a latin alphabet
\usepackage[T1]{fontenc}

% enhanced input encoding handling - utf8 for äÄüÜöÖß...
\usepackage[utf8]{inputenc}

%de­fines Adobe Times Ro­man as de­fault text font
\usepackage{mathptmx}
\usepackage{times} % needed for acronym package

%PDF linking package
\usepackage[hidelinks]{hyperref}


% Language Setup
\usepackage[\docLanguage]{babel}
% after babel - set chapter string
\AtBeginDocument{\renewcommand{\chaptername}{}}

% language specific bibliography style
\usepackage[numbers, square]{natbib}
%\setcitestyle{square,aysep={},yysep={;}}
\usepackage{babelbib}
% bliographystyle setup
% babel specific: babplain, babplai3, babalpha, babunsrt, bababbrv, bababbr3
\bibliographystyle{babunsrt}


% enumeration
\usepackage{enumitem}
% tabular extension tabularx
\usepackage{tabularx}

% math packages
\usepackage{amsmath}
\usepackage{nicefrac}
\usepackage{amsthm}
\usepackage{amsbsy}
\usepackage{amssymb}
\usepackage{amsfonts}
%\usepackage{MnSymbol}


%special characters
\usepackage{amssymb}
\usepackage{upgreek,textgreek}

% acronym package
\usepackage[printonlyused, footnote]{acronym}

% breakable text in \seqsplit{}
\usepackage{seqsplit}

% \textmu
\usepackage{textcomp}

% package provides a way to compile sections of a document using the same preamble as the main document
\usepackage{subfiles}

% driver-independent color extension - used by listings,tabularx
\usepackage[usenames,dvipsnames,table,xcdraw]{xcolor}

% -- SYNTAX HIGHLIGHTING --
\usepackage{listings}
%% bash command line Syntax Highlighting
\lstdefinestyle{BASH_CMD}{ 
  columns=fullflexible,            % copy pasteable listings
  language=bash,
  basicstyle=\small\sffamily,
  basicstyle   = \small \ttfamily,
  keywordstyle = [1]\small \ttfamily,
  keywordstyle = [2]\small \ttfamily,
  commentstyle = \small \ttfamily,
  numbers=none,
  captionpos=b, 
  breaklines=true,
  numberstyle=\tiny,
  numbersep=3pt,
  frame=tlrb,
  columns=fullflexible,
  backgroundcolor=\color{white!20},
  linewidth=\linewidth,
  literate=                        % replace in code
     {Ö}{{\"O}}1
     {Ä}{{\"A}}1
     {Ü}{{\"U}}1
     {ß}{{\ss}}2
     {ü}{{\"u}}1
     {ä}{{\"a}}1
     {ö}{{\"o}}1
}
 % adds style BASH_CMD
%\input{cfgs/listings/listings_def_lang_bash-script.tex} % adds style BASH_SCRIPT
\input{cfgs/listings/listings_def_lang_latex.tex} % adds style LATEX
%\input{cfgs/listings/listings_def_lang_matlab.tex} % adds style MATLAB
\input{cfgs/listings/listings_def_lang_python.tex} % adds style PYTHON
%\input{cfgs/listings/listings_def_lang_c++.tex} % adds style CPP
%\input{cfgs/listings/listings_def_lang_c.tex} % adds style C
%\input{cfgs/listings/listings_def_lang_json.tex} % adds style JSON

% HEADLINE CFG
\usepackage{fancyhdr} % Headers and footers
\usepackage{lastpage}
\usepackage{ifthen}
\setlength{\headheight}{1.5cm}
%\pagestyle{fancy} % All pages have headers and footers
% override plain page style for \part, \chapter or
% \maketitle, which implicit specifies plain page style
\input{cfgs/fancyhdr/fancyhdr_pagestyle_plain.tex}
% set list pagestyle
\input{cfgs/fancyhdr/fancyhdr_pagestyle_preface.tex}
% set default pagestyle
\input{cfgs/fancyhdr/fancyhdr_pagestyle_default_onepage.tex}
%\input{cfgs/fancyhdr/fancyhdr_pagestyle_default_twopage.tex}

\renewcommand{\chaptermark}[1]{\markright{#1}{}}
\renewcommand{\sectionmark}[1]{\markright{#1}{}}
\renewcommand{\headrulewidth}{0pt}
\renewcommand{\footrulewidth}{0pt}

% PICTURE CFG
\usepackage{verbatim}
\usepackage{graphicx}
\usepackage{epstopdf}
\usepackage{caption}
\usepackage[list=true,listformat=simple]{subcaption}
% floating prevention packages
\usepackage{float}    % used with [H] positioning parameter
\usepackage{placeins} % \FloatBarrier
% tikz packages
\usepackage{tikz}
\usepackage{standalone}
\usepackage{pgfplots}


% include only specified tex files - uncommend here
\includeonly{preface/cover,
             preface/abstract,
             preface/tableofcontents,
             preface/listoffigures,
             preface/listoftables,
             preface/lstlistoflistings,
             appendix/bibliography}

%-------------------
%╔═╗╔╦╗╦═╗╦╔╗╔╔═╗╔═╗
%╚═╗ ║ ╠╦╝║║║║║ ╦╚═╗
%╚═╝ ╩ ╩╚═╩╝╚╝╚═╝╚═╝
%-------------------
\newcommand{\strLecture}{Signale, Systeme und Sensoren}
\newcommand{\strDate}{\today}
\newcommand{\strAuthorA}{D. Wollmann}
\newcommand{\strAuthorB}{V. Bratulescu}
%\newcommand{\strAuthorC}{C. Author}
\newcommand{\strAuthorAEmail}{da161wol@htwg-konstanz.de}
\newcommand{\strAuthorBEmail}{vl161bra@htwg-konstanz.de}
%\newcommand{\strAuthorCEmail}{cauthor@htwg-konstanz.de}
% Versuchsbeschreibung
\newcommand{\strTopic}{Fourieranalyse und Akustik}
\newcommand{\strAbstract}{

In diesem Versuch wird die Fourieranalyse auf einen akustischen Ton angewendet. Dieser wird in Form eines Keyboards produziert und mithilfe von Python verarbeitet. Dabei wird die Technik der Fouriertransformation angewendet.
}
% hyperref customization
\hypersetup{
	pdftitle     = {\strTopic}, % title
	pdfsubject   = {\strLecture}, % subject of the document
	pdfauthor    = {\strAuthorA, \strAuthorB}, % author
	pdfkeywords  = {}, % list of keywords
	pdfcreator   = {}, % creator of the document
	pdfproducer  = {}, % producer of the document
	colorlinks   = false, % false: boxed links; true: colored links
	linkcolor    = red, % color of internal links (change box color with linkbordercolor)
    citecolor    = green, % color of links to bibliography
    filecolor    = magenta, % color of file links
    urlcolor     = cyan, % color of external links
	%bookmarks    = true, % show bookmarks bar?
	unicode	     = true, % non-Latin characters in Acrobat’s bookmarks
	pdftoolbar   = true, % show Acrobat’s toolbar?
	pdfmenubar   = true, % show Acrobat’s menu?
    pdffitwindow = false, % window fit to page when opened
	pdfnewwindow = true % links in new PDF window
}

%-----------------------------------------
% ╔╗ ╔═╗╔═╗╦╔╗╔  ╔╦╗╔═╗╔═╗╦ ╦╔╦╗╔═╗╔╗╔╔╦╗
% ╠╩╗║╣ ║ ╦║║║║   ║║║ ║║  ║ ║║║║║╣ ║║║ ║
% ╚═╝╚═╝╚═╝╩╝╚╝  ═╩╝╚═╝╚═╝╚═╝╩ ╩╚═╝╝╚╝ ╩
%-----------------------------------------

\begin{document}
\pagenumbering{Roman}

\setcounter{section}{0}
\include{preface/cover}


\begin{center}
{\Large \textbf{Zusammenfassung (Abstract)}}
\end{center}

\bigskip

\begin{center}
	\begin{tabular}{p{2.8cm}p{5cm}p{5cm}}
		Thema: & \multicolumn{2}{p{10cm}}{\raggedright\strTopic} \\
		 & & \\
		Autoren: & \strAuthorA & \href{mailto:\strAuthorAEmail}{\strAuthorAEmail} \\
		 & \strAuthorB & \href{mailto:\strAuthorBEmail}{\strAuthorBEmail} \\
		 & & \\
		Betreuer: & Prof. Dr. Matthias O. Franz & \href{mailto:mfranz@htwg-konstanz.de}{mfranz@htwg-konstanz.de} \\
		 &  Jürgen Keppler & \href{mailto:juergen.keppler@htwg-konstanz.de}{juergen.keppler@htwg-konstanz.de} \\
		 &  Martin Miller & \href{mailto:martin.miller@htwg-konstanz.de}{martin.miller@htwg-konstanz.de} \\
	\end{tabular}
\end{center}

\bigskip

\noindent
\strAbstract

\thispagestyle{lists}



\clearpage

%
% TABLE OF CONTENTS
%
\thispagestyle{lists}
%
% TABLE OF CONTENTS
%
\tableofcontents
\thispagestyle{plain}
\newpage

%
% Abbildungsverzeichnis
%
%
% Abbildungsverzeichnis
%
\phantomsection
\addcontentsline{toc}{chapter}{Abbildungsverzeichnis}
\listoffigures
\thispagestyle{lists}
\newpage

%
% Tabellenverzeichnis
%
%
% Tabellenverzeichnis
%
\phantomsection
\addcontentsline{toc}{chapter}{Tabellenverzeichnis}
\listoftables
\thispagestyle{lists}
\newpage

%
% Listingverzeichnis
%
%
% Listingverzeichnis
%
\phantomsection
\renewcommand\lstlistingname{Listing}
\renewcommand\lstlistlistingname{Listingverzeichnis}
\lstlistoflistings
\addcontentsline{toc}{chapter}{Listingverzeichnis}
\thispagestyle{lists}
\newpage


%--------------------------
% ╔═╗╦ ╦╔═╗╔═╗╔╦╗╔═╗╦═╗╔═╗
% ║  ╠═╣╠═╣╠═╝ ║ ║╣ ╠╦╝╚═╗
% ╚═╝╩ ╩╩ ╩╩   ╩ ╚═╝╩╚═╚═╝
%--------------------------

\pagenumbering{arabic}
\setcounter{page}{1}
\pagestyle{default}
%
% CHAPTER Einleitung
%
\chapter{Einleitung}
\label{chap:EINL}

In dem dritten Versuch wird die Fourieranalyse auf ein akustisches Signal angewendet. Das akustische Signal wird durch ein Keyboard erzeugt, welches einen konstanten Ton abspielt, der periodisch ist. Der abgespielte Ton mithilfe der pyaudio Bibliothek in Python aufgenommen und weiterverarbeitet. Dabei werden die Grundperiode, Signaldauer und weitere Merkmale berechnet. Anschließend wird durch die Fouriertransformation das Amplitudenspektrum berechnet und graphisch dargestellt.

%
% CHAPTER Versuch 1
%
\chapter{Versuch 1: Bestimmung der Tonhöhe eines akustischen Signals}
\label{chap:VERSUCH_1}

\section{Fragestellung, Messprinzip, Aufbau, Messmittel}
\label{chap:VERSUCH_1_FRAGESTELLUNG}

In diesem Versuch wird die Fourieranalyse auf ein akustisches Signal angewendet. Dazu wird mit einem Keyboard ein konstanter Ton (C4) erzeugt und mittels eines Mikrofons aufgenommen. Der aufgenommene Ton soll eine genügend hohe Amplitude haben und gleichmäßig wiederholende Perioden aufweisen. Der Ton wird über das Mikrofon mittels der pyaudio Bibliothek aufgenommen und anschließen in einer .csv Datei abgespeichert.
Das verwendete Mikrofon wurde in der Nähe der Ausgabelautsprecher des Keyboards platziert, sodass der abgespielte Ton gut aufgenommen werden kann.

\begin{figure}[H]
	\centering\small
	\includegraphics[width=0.5\textwidth]{media/Versuchsaufbau.png}
	\caption{Versuchsaufbau}
	\label{fig:VERSUCH_1_VERSUCHSAUFBAU}
\end{figure}

Des Weiteren ist die Aufgabe mehrere Perioden des Signals graphisch darzustellen und anhand dieses Plots die Grundfrequenz und Grundperiode zu ermitteln. Zu dem soll die Signaldauer, Abtastfrequenz, Signallänge und Abtastintervall berechnet werden.
Im Anschluss soll die Fouriertransformation mithilfe der Python Funktion numpy.fft.fft() angewendet und daraus das Amplitudensprektrum bestimmt und graphisch dargestellt werden. Dabei ist zu beachten, dass die x-Achse mit der Frequenz nicht in Hertz sondern in Anzahl Schwingungen innerhalb der gesamten Signaldauer definiert ist. Die zugehörige Frequenz f in Hertz kann jedoch durch die Formel 

\begin{equation*}
	f = \frac{n}{M \cdot \Delta t}
\end{equation*}
berechnet werden. Durch das Amplitudenspektrum kann anschließend die ursprüngliche Grundfrequenz identifiziert werden.
\newpage
\section{Messwerte}
\label{chap:VERSUCH_1_MESSWERTE}
In Abbildung \ref{fig:VERSUCH_1_PERIODEN} ist ein Ausschnitt einzelner Perioden des aufgenommenen Signals dargestellt, welches wir durch ein Keyboard erzeugt haben. Die x-Achse represäntiert den Zeitpunkt in Millisekunden von der Aufnahme der jeweiligen Messung. Die jeweiligen Zeitpunkte haben wir berechnet und den Daten in der .csv Datei hinzugefügt.
Um den jeweiligen Zeitpunkt zu berechnen, haben wir das Abtastintervall mit dem jeweiligen Index des Messwertes multipliziert.

\begin{figure}[H]
	\centering\small
	\includegraphics[width=0.5\textwidth]{media/plotDecodedRichtig.png}
	\caption{Ausschnitt vom Signal des Keyboards}
	\label{fig:VERSUCH_1_PERIODEN}
\end{figure}

In der nachfolgenden Abbildung \ref{fig:VERSUCH_1_DR} ist das gesamte aufgenommene Signal zu sehen. Da wir den Ton schon vor Beginn der Aufnahme abgespielt haben, ist kein Einschwingvorgang oder ähnliches zu erkennen.

\begin{figure}[H]
	\centering\small
	\includegraphics[width=0.5\textwidth]{media/decodedRichtig.png}
	\caption{Gesamtes Amplitudenspektrum bis zur Hälfte}
	\label{fig:VERSUCH_1_DR}
\end{figure}
\newpage
\section{Auswertung}
\label{chap:VERSUCH_1_AUSWERTUNG}
Aus der Abbildung \ref{fig:VERSUCH_1_PERIODEN} kann die Grundperiode in Millisekunden abgelesen werden, welche 2ms beträgt. Dadurch kann die Grundfrequenz von 500Hz abgeleitet werden. Die Signaldauer, Abtastfrequenz, Signallänge und Abtastintervall können durch einfache Berechnungen ermittelt werden. Die Ergebnisse sind in der Tabelle \ref{fig:VERSUCH_1_AUSWERTUNG_TABELLE} zu sehen.

\begin{table}[H]
\center
\begin{tabular}{|l|l|}
\hline
\multicolumn{1}{|c|}{\textbf{Beschreibung}}   & \textbf{Wert} \\ \hline
Grundperiode {[}ms{]}                         & 2          \\ \hline
Grundfrequenz {[}Hz{]}                        & 500        \\ \hline
Signaldauer {[}sec{]}                         & 5.171     \\ \hline
Abtastfrequenz {[}Hz{]}                       & 43558.89   \\ \hline
Signallänge {[}Anzahl der Abtastzeitpunkte{]} & 225280        \\ \hline
Abtastintervall {[}sec{]}                     & 2.295e-05  \\ \hline
\end{tabular}
\caption{Ergebnisse}
\label{fig:VERSUCH_1_AUSWERTUNG_TABELLE}
\end{table}

Das Amplitudenspektrum wird durch den Betrag der numpy Funktion np.fft.fft() berechnet. Wie zuvor erwähnt ist die x-Achse Beschriftung jedoch in der Einheit Anzahl Schwingungen innerhalb der gesamten Signaldauer und nicht in Hertz eingegeben. Deshalb muss die oben genannte Formel angewandt werden. Außerdem wird das Amplitudenspektrum nur bis zur Hälfte angezeigt. Dieses wird in Abbildung \ref{fig:VERSUCH_1_ASB} dargestellt.

\begin{figure}[H]
	\centering\small
	\includegraphics[width=0.5\textwidth]{media/AmplitudenspektrumBreit.png}
	\caption{Gesamtes Amplitudenspektrum bis zur Hälfte}
	\label{fig:VERSUCH_1_ASB}
\end{figure}

In Abbildung \ref{fig:VERSUCH_1_ASS} ist ein Ausschnitt des Amplitudenspektrums zu sehen.

\begin{figure}[H]
	\centering\small
	\includegraphics[width=0.5\textwidth]{media/AmplitudenspektrumSchmal.png}
	\caption{Ausschnitt des Amplitudenspektrums}
	\label{fig:VERSUCH_1_ASS}
\end{figure}

Der größte Amplitudenausschlag innerhalb des Spektrums liegt bei der Frequenz 1029Hz. Außerdem ist die Wellenzahl k zu berechnen, die wir mit der Formel 

\begin{equation*}
	k = \frac{1}{\lambda}
\end{equation*}
ermittelt haben. Wobei $\lambda$ die Wellenlänge ist, die in unserem Fall 2ms beträgt. Dies kann in der Abbildung \ref{fig:VERSUCH_1_PERIODEN} abgelesen werden. Dadurch ergibt sich die Wellenzahl von 500Hz.


Die Frequenz mit dem maximalen Amplitudenwert kann mithilfe einer einfachen for Schleife, die über das Spektrum iteriert, ermittelt werden.

\begin{table}[H]
\center
\begin{tabular}{|l|l|}
\hline
\multicolumn{1}{|c|}{\textbf{Beschreibung}}   & \textbf{Wert} \\ \hline
Maximaler Amplitudenausschlag                         & 191860000.7          \\ \hline
Frequenz mit der größten Amplitude {[}Hz{]}                        & 1029.22        \\ \hline
\end{tabular}
\caption{Ergebnisse}
\label{fig:VERSUCH_1_MAX}
\end{table}
\newpage
\section{Interpretation}
\label{chap:VERSUCH_1_INTERPRETATION}
Wie zu erwarten, ist in Abbildung \ref{fig:VERSUCH_1_PERIODEN} sehr gut ersichtlich, dass der Ton gleichmäßig und mit konstanter Tonhöhe gespielt wurde. 

Aus der Tabelle \ref{fig:VERSUCH_1_AUSWERTUNG_TABELLE} lässt sich schließen, dass die Signaldauer das Ergebnis der Multiplikation von der Signallänge mit dem Abstastintervall ist. Des Weiteren ist das Abtastintervall komplementär zur Abtrastfrequenz, da die Frequenz zunimmt, wenn das Intervall abnimmt. Analog gilt das auch für die Grundperiode und Grundfrequenz.

In Abbildung \ref{fig:VERSUCH_1_ASS} sind die harmonischen vielfachen der Grundfrequenz sehr deutlich zu sehen.
Des Weiteren ist sehr schön zu sehen, dass nicht so viele Obertöne vorhanden sind, was darauf zurückzuführen ist, dass der Ton durch ein digitales Keyboard erzeugt wurde und nicht durch ein klassisches Klavier.

%
% CHAPTER Anhang
%
\renewcommand\thesection{A.\arabic{section}}
\renewcommand\thesubsection{\thesection.\arabic{subsection}}

\chapter*{Anhang}
\label{chap:APPENDIX}
\addcontentsline{toc}{chapter}{Anhang}
%\setcounter{chapter}{0}
\addtocounter{chapter}{1}
\setcounter{section}{0}

\section{Quellcode}
\label{chap:APPENDIX_SOURCECODE}

\subsection{Quellcode Versuch 1}
\label{chap:APPENDIX_SOURCECODE_V1a}
\lstinputlisting[style=PYTHON, frame=single, caption=Skript Versuch 1a, captionpos=b, label=lst:APPENDIX_SOURCECODE_PLOT]{code/Aufgabe1.py}
\label{chap:APPENDIX_SOURCECODE_V1b}
\lstinputlisting[style=PYTHON, frame=single, caption=Skript Versuch 1b, captionpos=b, label=lst:APPENDIX_SOURCECODE_PLOT2]{code/Aufgabe1.2.py}
\label{chap:APPENDIX_SOURCECODE_V1c}
\lstinputlisting[style=PYTHON, frame=single, caption=Skript Versuch 1c, captionpos=b, label=lst:APPENDIX_SOURCECODE_PLOT3]{code/Aufgabe1.3.py}

%
% Literaturverzeichnis
%

\end{document}
%------------------------------------
% ╔═╗╔╗╔╔╦╗  ╔╦╗╔═╗╔═╗╦ ╦╔╦╗╔═╗╔╗╔╔╦╗
% ║╣ ║║║ ║║   ║║║ ║║  ║ ║║║║║╣ ║║║ ║
% ╚═╝╝╚╝═╩╝  ═╩╝╚═╝╚═╝╚═╝╩ ╩╚═╝╝╚╝ ╩
%------------------------------------