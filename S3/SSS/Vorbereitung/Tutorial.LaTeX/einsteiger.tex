\documentclass[11pt]{scrartcl}
\usepackage{ucs}
\usepackage[utf8x]{inputenc}
\usepackage[T1]{fontenc}
\usepackage[ngerman]{babel}
\usepackage{amsmath,amssymb,amstext}
\usepackage{graphicx}
\usepackage[automark]{scrpage2}
\pagestyle{scrheadings}
\clearscrheadfoot
\ifoot[]{\author}
\ofoot[]{\pagemark} 

\title{Märchen und andere Geschichten}
\author{Max Mustermann}
\date{\today{}, Graz}

\begin{document}
\maketitle
\tableofcontents

\section{Einleitende Worte}
\label{sec:einleitende-worte}

Finster war's, der Mond schien helle auf die grünbeschneite Flur, als
ein Wagen blitzesschnelle langsam um die runde Ecke fuhr. Drinnen
saßen stehend Leute schweigend ins Gespräch vertieft, als ein
totgeschossner Hase auf dem Wasser Schlittschuh lief und ein
blondgelockter Knabe mit kohlrabenschwarzem Haar auf die grüne Bank
sich setzte, die gelb angestrichen war.

Alice kann es einfach nicht lassen, sie muß dem weißen Kaninchen mit
der großen Uhr folgen und landet prompt im Wunderland. Auf ihrer Reise
durch diese fröhlich bunte, aber auch sehr eigenartige Welt begegnet
sie einer gestiefelten Raupe, dem verrückten Hutmacher und ist zu Gast
bei einer nicht Geburtstags-Party. Einer hinterlistigen Tigerkatze hat
es das Mädchen schließlich zu verdanken, daß sie den Zorn der
Herz-Königin auf sich zieht. So etwas kann einem eigentlich nur im
Traum passieren.

\begin{center}
	Text in dieser Umgebung wird zentriert dargestellt.
\end{center}

\begin{itemize}
 	\item Alice im Wunderland
 	\item Till Eulenspiegel
 	\item Harry Potter
 	\begin{enumerate}
    		\item Der Stein der Weisen
    		\item Kammer des Schreckens
    		\item Der Gefangene von Askaban
    		\item Der Feuerkelch
    		\item Der Orden des Phönix
  	\end{enumerate}
  	\item Jim Knopf
\end{itemize}

\section{Mathematik}
\label{sec:mathematik}
 
\subsection{Unterstufe}
\label{sec:unterstufe}
\begin{equation*}
 	a_{ij} + a_2 = 0
\end{equation*}
\begin{equation*}
  	\frac{1}{a} + \frac{1}{b} = \frac{a+b}{ab}
\end{equation*}
\begin{equation*}
  	\sigma + \tau = \alpha
\end{equation*}
 
\subsection{Oberstufe} 
\label{sec:oberstufe}
\begin{equation}
 	\label{eq:1}
 	\left( \frac{a}{b} \right)' = \frac{a'b-ab'}{b^{2}}
\end{equation} 
Es gilt die Invariante $b \neq 0$.
\begin{equation}
 	\label{eq:2}
 	\int\limits_{a}^{b} x^{2} \, dx = \frac{ b^{3} - a^{3} }{3}
\end{equation}

\begin{equation}
	\label{eq:3}
	c = \sqrt{ a^{2} + b^{2} }
\end{equation}

\section{Fortgeschrittene Anwendung}
\label{sec:fortg-anwend}
 
\subsection{Was macht Alice im Wunderland?}
\label{sec:was-macht-alice}
In Abschnitt \ref{sec:einleitende-worte} wurde ein Mädchen namens
Alice erwähnt. Was sie im Wunderland erlebt, kann in einem Buch
nachgelesen werden.
 
\subsection{Analyse}
\label{sec:analyse}
Die Gleichungen \eqref{eq:1} bis \eqref{eq:3} beherrschen wir bestens.
Alice, von der wir auf Seite \pageref{sec:einleitende-worte} gehört
haben, kennt diese Gleichungen wahrscheinlich nicht.

\begin{center}
 	\includegraphics[width=0.1\textwidth]{TUG-logo}
\end{center}
Das Bild zeigt unser Logo\footnote{Bitte korrekt verwenden.}.
	
\end{document}

Wird nicht angezeigt!